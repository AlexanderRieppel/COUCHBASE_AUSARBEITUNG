\documentclass[a4paper,12pt]{scrreprt}
\usepackage[T1]{fontenc}
\usepackage[utf8]{inputenc}
\usepackage[ngerman]{babel}
\usepackage[table]{xcolor}% http://ctan.org/pkg/xcolor
\usepackage{tabu}
\usepackage{graphicx}
\usepackage{lmodern}
\usepackage{hyperref}

\begin{document}


%\titlehead{Kopf} %Optionale Kopfzeile
\author{Alexander Rieppel} %Zwei Autoren
\title{Datenbanksynchronisation mit Couchbase und Couchbase Lite} %Titel/Thema
\subject{VSDB} %Fach
\subtitle{Ausarbeitung} %Genaueres Thema, Optional
\date{\today} %Datum
\publishers{5AHITT} %Klasse

\maketitle
\tableofcontents
\bibliographystyle{alphadin} 

\chapter{Einf"uhrung}
\section{Datenbanksynchronisation}
\section{M"oglichkeiten}
Wie bereits erw"ahnt ist Datenbanksynchronisation in verschiedensten Bereichen sinnvoll. Anschlie"send ein Überblick von bestehenden L"osungen um g"angige Datenbanken zu Synchronisieren:
\begin{itemize}
\item MySQL und SQLite
\end{itemize}
\section{Umsetzung im Diplomprojekt}

\chapter{Couchbase Lite}
Couchbase Lite ist eine leichtgewichtige, dokumenten-orientierte und leicht synchronisierbare Datenbank welche speziell f"ur den Einsatz in mobilen Anwendungen und Ger"aten geeignet ist.\\\\  Leichtgewichtig bedeutet:

\begin{itemize}
\item Die Datenbank Engine ist eine in der Applikation gebundene Bibliothek und kein separater Serverprozess.
\item Sie besitzt sehr klein gehaltenen Code damit Apps die auf die Schnittstelle zur"uckgreifen rasch heruntergeladen werden k"onnen.
\item Garantiert eine kurze Startzeit da mobile Ger"ate meist geringere CPU-Leistung haben als PCs.
\item Mobile Datens"atze sind zwar relativ klein, allerdings k"onnen manche Dokumente gro"se multimediale Anh"ange haben, weswegen ein geringe Speicherauslastung von N"oten ist.
\item Bietet auch eine gute Performance, wobei diese zu einem gro"sen Teil von der implementierten Applikation abh"angt.
\end{itemize}
Dokumenten-orientiert bedeutet:

\begin{itemize}
\item Speichert Eintr"age im flexiblen JSON-Format, weshalb keine vordefinierten Schemata ben"otigt werden.
\item Dokumente k"onnen eine frei w"ahlbare Gr"o"se von Binary-Anh"angen besitzen, wie z.B. multimediale Inhalte.
\item Das Datenformat der Applikation kann sich "uber die Zeit weiterentwickeln, ohne dass die Datenbank ge"andert werden muss.
\item MapReduce Indizierung bietet eine schnelle Datensatzabfrage, ohne dass spezielle Query-Languages verwendet werden m"ussen.
\end{itemize}
Leicht synchronisierbar bedeutet:
\begin{itemize}
\item Zwei Kopien einer Datenbank k"onnen problemlos "uber einen Replikations-Algorithmus synchronisiert werden.
\item Die Synchronisation kann On-Demand oder fortlaufend stattfinden.
\item Ger"ate k"onnen auch nur Teilmengen einer riesigen Datenbasis eines Remote-Servers synchronisieren.
\item Die Sync-Engine erlaubt auch das Synchronisieren "uber unbest"andige und unzuverl"assige Netzwerkverbindungen.
\item Konflikte k"onnen einfach "uber einen Merge-Algorithmus, gefunden und behoben werden.
\item Revisions-B"aume erlauben auch komplexe Replikations-Topologien, wie z.B. Server-to-Server (f"ur mehrere Datenzentren) und Peer-to-Peer, ohne Datenverlust oder Konflikte bef"urchten zu m"ussen.
\end{itemize}
\cite{couch1}
\section{Synchronisation mit Couchbase Sync Gateway}
\section{Vorteile dieser Umsetzung}
\section{Fazit}
\bibliography{ref}
\end{document}